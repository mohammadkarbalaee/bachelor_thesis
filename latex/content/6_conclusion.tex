\chapter{Conclusions\markboth{Conclusions}{}}

This thesis has successfully explored the enhancement of traffic sign recognition (TSR) systems through the integration of collaborative perception in Vehicular Ad Hoc Networks (VANETs) under adverse conditions. By leveraging the strengths of deep learning models, specifically the MobileNetV2 architecture, and the real-time data-sharing capabilities of VANETs, this study addressed key challenges associated with TSR in dynamic and unpredictable driving environments.

The experimental results demonstrated that while the standalone TSR model performed well under normal conditions, its accuracy significantly dropped in the presence of environmental distortions such as motion blur and rotational variations. However, the integration of VANET-based collaborative perception notably improved recognition accuracy, particularly under challenging conditions. The consensus mechanism employed in the VANET simulation allowed vehicles to cross-verify recognition results, thereby mitigating individual model inaccuracies and enhancing overall system reliability. This improvement was most evident in scenarios involving rotational distortions, where the collaborative system outperformed the standalone model by a substantial margin.

The findings of this research emphasize the practical viability of integrating TSR systems with VANETs to enhance autonomous vehicle performance. The collaborative approach not only improves recognition accuracy but also contributes to safer and more efficient transportation systems. These results underscore the importance of real-time vehicular communication in overcoming the limitations of traditional TSR systems, especially in complex real-world scenarios.

Future research should focus on optimizing the communication protocols within VANETs to handle high-density traffic and ensure scalability for broader deployment. Additionally, exploring the integration of more advanced machine learning models and multi-sensor data fusion techniques could further enhance the robustness and adaptability of TSR systems. Investigating cybersecurity measures to safeguard data integrity in VANET communications will also be crucial for the safe implementation of these systems.

In conclusion, this study provides a significant step forward in improving TSR systems by combining deep learning and collaborative vehicular communication. The positive results affirm the potential of this integrated approach to contribute to the development of safer and more reliable intelligent transportation systems.

