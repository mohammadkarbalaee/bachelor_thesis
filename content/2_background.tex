\chapter{Background\markboth{Background}{}}

\section{Vehicle-to-Everything (V2X) Communication}

Vehicle-to-Everything (V2X) communication is a groundbreaking technology that enables vehicles to exchange data with
 their surroundings, including other vehicles, infrastructure, pedestrians, and cloud-based systems.
  This interconnected framework is a cornerstone of modern \ac{ITS}, designed to enhance road safety, improve traffic flow, and facilitate autonomous driving.

\subsection{Types of V2X Communication}
V2X encompasses several key components. \ac{V2V} communication allows direct data exchange between vehicles, 
enabling applications such as collision avoidance and coordinated lane changes. \ac{V2I} 
extends this interaction to roadside elements like traffic lights and road sensors, which provide vehicles with vital 
updates about traffic conditions or hazards. Additionally, \ac{V2P} communication ensures vehicles
 are aware of nearby pedestrians, even in scenarios with poor visibility. Finally, \ac{V2C} links vehicles 
 to cloud servers for updates on navigation, weather, or software improvements. \cite{8605302}

 \begin{figure}[H]
    \centering
    \includegraphics[width=1\textwidth]{images/figure1.png}
    \caption{An overview of V2X scenario}
    \label{fig:fig1}
\end{figure}

Figure~\ref{fig:fig1} illustrates a V2X communication network in a smart city environment, showcasing the interactions between
 vehicles, infrastructure, pedestrians, and networks. Various types of V2X communication are represented:
  \ac{V2N} connects vehicles to cloud-based systems via the Cellular \ac{eNB},
   which serves as the backbone of the cellular communication infrastructure. The Cellular eNB provides real-time updates
    and broad connectivity by leveraging 4G LTE and 5G technologies, enabling vehicles to access services such as navigation, 
    traffic information, and emergency alerts. \cite{cite-key}

Vehicle-to-Infrastructure (V2I) is enabled through \ac{RSUs}, which are positioned near roadways 
and intersections. RSUs act as intermediaries between vehicles and the infrastructure, collecting and disseminating
 localized traffic information such as signal timings, road hazards, or construction updates. These units enhance 
 traffic management and safety by maintaining a continuous flow of communication with nearby vehicles and infrastructure 
 elements like traffic lights and road signs. \cite{10806826}
    

 \subsection{Technologies Enabling V2X Communication}

 The technology behind V2X is built on two major standards. \ac{DSRC},
 a Wi-Fi-based protocol, is optimized for low-latency, reliable communication, making it suitable for 
 safety-critical applications like emergency braking. \ac{C-V2X}, on the other hand, leverages 
 4G LTE and 5G networks to support broader connectivity, enabling advanced functionalities such as real-time updates and 
 large-scale data sharing.

 \subsection{Applications of V2X Communication}

Applications of V2X are vast and transformative. In addition to enhancing safety through 
collision prevention, V2X optimizes traffic management by reducing congestion and enabling 
efficient vehicle platooning. For autonomous vehicles, V2X complements onboard sensors like 
cameras and LiDAR, providing an additional layer of environmental awareness.

\subsection{Challenges in V2X Communication}

Despite its potential, V2X faces several challenges. Security and privacy concerns arise from the constant
 exchange of real-time data, while ensuring seamless interoperability across manufacturers remains a significant 
 hurdle. Furthermore, achieving the ultra-low latency required for critical safety applications and deploying 
 infrastructure in less-developed areas are ongoing obstacles.

 \subsection{V2X in Traffic Sign Recognition}

In the context of traffic sign recognition (TSR), V2X offers unique opportunities for enhancing system reliability.
 Through collaborative perception, vehicles can share recognition data in real-time, allowing a consensus mechanism 
 to validate and improve recognition accuracy. This integration addresses common challenges in TSR, such as occlusion, 
 adverse weather conditions, and adversarial attacks, laying the foundation for a more robust and trustworthy recognition 
 system—a central focus of this study.